\documentclass[11]{article}
\usepackage[top=2cm,bottom=2cm,left=2cm,right=2cm]{geometry}
\newcommand\tab[1][1cm]{\hspace*{#1}}
\usepackage{amsmath}
\usepackage[makeroom]{cancel}
\usepackage{graphicx}
\usepackage{multicol}
\newcommand\coord[3]{\begin{pmatrix} #1 \\#2 \\#3 \end{pmatrix}}
\newcommand\coordo[2]{\begin{pmatrix} #1 \\#2 \end{pmatrix}}
\newcommand{\CQFD}{\hfill {\bfseries CQFD}}
\usepackage{color}
\setlength{\fboxsep}{2.5mm}
\setlength{\fboxrule}{0.25mm}
\usepackage{amssymb}
\usepackage{tkz-tab}
\usepackage[normalem]{ulem}
\usepackage{soul}
\usepackage{fancyhdr}
\pagestyle{fancy}
\usepackage{ulem}
\renewcommand\headrulewidth{1pt}
\fancyhead[L]{Cours : les transformations dans un espace euclidien }
\fancyhead[R]{Erwan Kerbrat}
\newcommand{\soulignerr}[1]{\color{red}{\uline{\color{black}{ #1} }}\color{black}}
\usepackage{mathtools}
\newcommand\myeq{\stackrel{\mathclap{\normalfont\mbox{\begin{tiny} $x\in V_a$\end{tiny} }}}{\approx}}
\newcommand\mye{\stackrel{\mathclap{\normalfont\mbox{\begin{tiny} $(x;y)\in V_{(a;b)}$\end{tiny} }}}{\approx}}
\setlength\parindent{0pt}
\begin{document}


\begin{center}
\begin{Large}
Les transformations dans un espace euclidien \\
\end{Large}
\end{center}
\begin{center}
\begin{Large}
Rotation, translation, symétrie \\
\end{Large}
\end{center}


\hrulefill \\

\begin{enumerate}
\begin{Large} \item La rotation \end{Large} \\

\begin{enumerate}
\item Contexte : \\
\begin{itemize}
\item On a deux repères : $R_1:(0_1;\overrightarrow{x_1};\overrightarrow{y_1})$ et $R_2:(0_1;\overrightarrow{x_2};\overrightarrow{y_2})$. \\
\includegraphics[width=3in]{image}
\item On a les coordonnées d'un point $M$ (ou d'un vecteur $\overrightarrow{OM}$) dans le repère 1 : \\ $M\coordo{x}{y}_{R_1}$ (ou $\overrightarrow{O_1M}\coordo{x}{y}_{R_1}$).
\item On veut trouver les coordonnées de $M\coordo{X}{Y}_{R_2}$ (ou de $\overrightarrow{O_1M}\coordo{X}{Y}_{R_2}$) connaissant l'angle $\theta:(\overrightarrow{x_1};\overrightarrow{x_2})$
\end{itemize}

\item Explications : \\

\begin{tabular}{|c|c|c|c|}
\hline
\begin{minipage}{11em}
\vspace{1\baselineskip}
\includegraphics[width=1.5in]{image} \\
$\overrightarrow{x_1}=cos(\theta)\overrightarrow{x_2}+sin(\theta)\overrightarrow{y_2}$ \\
\end{minipage} 
& 
\begin{minipage}{11em}
\vspace{1\baselineskip}
\includegraphics[width=1.5in]{image}\\
$\overrightarrow{y_1}=-sin(\theta)\overrightarrow{x_2}+cos(\theta)\overrightarrow{y_2}$ \\
\end{minipage} 
& 
\begin{minipage}{11em}
\vspace{1\baselineskip}
\includegraphics[width=1.5in]{image}\\
$\overrightarrow{x_1}=cos(\theta)\overrightarrow{x_2}+sin(\theta)\overrightarrow{y_2}$ \\
\end{minipage} 
& 
\begin{minipage}{11em}
\vspace{1\baselineskip}
\includegraphics[width=1.5in]{image}\\
$\overrightarrow{x_1}=cos(\theta)\overrightarrow{x_2}+sin(\theta)\overrightarrow{y_2}$ \\
\end{minipage}  \\
\hline
\end{tabular}
\begin{minipage}{30em}
\begin{align*}
\overrightarrow{OM} : \coordo{x}{y}_{R_1} & \Leftrightarrow \overrightarrow{OM}&&=x\overrightarrow{x_1}+y\overrightarrow{y_1} \\
&& &=x(cos(\theta)\overrightarrow{x_2} +sin(\theta)\overrightarrow{y_2}) +y(-sin(\theta)\overrightarrow{x_2}+cos(\theta)\overrightarrow{y_2}) \\
&& &=(x cos(\theta)-y sin(\theta))\overrightarrow{x_2}+(x sin(\theta)+y cos(\theta))\overrightarrow{y_2} \\
& &&\Leftrightarrow \overrightarrow{OM}:\coordo{x cos(\theta)-y sin(\theta)}{x sin(\theta) + y cos(\theta)}_{R_2} 
\end{align*}
\end{minipage}

\newpage

\item Exemple :  \\

\begin{itemize}
\item Si on vous donne $M\coordo{2}{2\sqrt{3}}_{R_1}$ (les coordonnées de $M$ dans le repère 1), vous pouvez donner les coordonnées de $M\coordo{X}{Y}_{R_2}$ (les coordonnées de $M$ dans le repère 2). \\ Avec $R_2$ le repère obtenu par rotation de $R_1$ autour de $O_1$ et d'angle $\theta=\frac{\pi}{6}$ \\

Ces coordonnées sont : 
\begin{itemize}
\item $X = (2)cos(\frac{\pi}{6})-(2\sqrt{3})sin(\frac{\pi}{6}) = (2)(\frac{\sqrt{3}}{2})-(2\sqrt{3})\frac{1}{2}=0$ 
\item $Y = (2)sin(\frac{\pi}{6})+(2\sqrt{3})cos(\frac{\pi}{6}) = (2)(\frac{1}{2})+(2\sqrt{3})\frac{\sqrt{3}}{2}=4$ 
\end{itemize} 
\vspace{1\baselineskip}

Illustration : \\
\includegraphics[width=2in]{image} \\
\end{itemize}


\item Matriciellement 

Plutôt que de faire les opérations \begin{minipage}{13em} $X=x cos(\theta) - y sin(\theta)$ \\ $Y=x sin(\theta) + y cos(\theta)$ \end{minipage}, on calculera le produit matriciel : 
$$\begin{pmatrix}
X \\ \\
Y
\end{pmatrix}_{R_2}   = 
\begin{pmatrix}
cos(\theta) & -sin(\theta) \\ \\
sin(\theta) & cos(\theta)
\end{pmatrix} \begin{pmatrix}
x \\ \\
y\end{pmatrix}_{R_1}
$$

Ou encore $V=R_{\theta}U$ avec $U = \begin{pmatrix} x \\ \\ y\end{pmatrix}_{R_1}$, $V = \begin{pmatrix} X \\ \\ Y \end{pmatrix}_{R_2}$ et $R_{\theta} = \begin{pmatrix}
cos(\theta) & -sin(\theta) \\ \\
sin(\theta) & cos(\theta)
\end{pmatrix} $. \\ \\

On appelle $R_{\theta}$ la matrice de rotation d'angle $\theta$. \\

Pour faire l'opération réciproque (passer des coordonnées de $R_2$ dans $R_1$), on peut multiplier un vecteur par la matrice inverse de $R_{\theta}$ : $R_{-\theta} : \begin{pmatrix}
cos(-\theta) & -sin(-\theta) \\ \\
sin(-\theta) & cos(-\theta)
\end{pmatrix} = \begin{pmatrix}
cos(\theta) & sin(\theta) \\ \\
-sin(\theta) & cos(\theta)
\end{pmatrix}$

\end{enumerate}

\newpage

\begin{Large} \item La translation \end{Large} \\
\begin{enumerate}
\item Contexte : 
\begin{itemize}
\item On a deux repères : $R_1:(0_1;\overrightarrow{x_1};\overrightarrow{y_1})$ et $R_2:(0_2;\overrightarrow{x_1};\overrightarrow{y_1})$. \\
\includegraphics[width=3in]{image}
\item On a les coordonnées d'un point $M$ (ou d'un vecteur $\overrightarrow{O_1M}$) dans le repère 1 :  $M\coordo{x}{y}_{R_1}$ .
\item On veut trouver les coordonnées de $M\coordo{X}{Y}_{R_2}$  connaissant  $\overrightarrow{O_1O_2}$.
\end{itemize}

\item Explications  : \\


\begin{multicols}{2}
\begin{minipage}{23em}
On a : 
\begin{itemize}
\item $\overrightarrow{O_1M}=x\overrightarrow{x_1}+y\overrightarrow{y_1}$
\item $\overrightarrow{O_2M}=X\overrightarrow{x_1}+Y\overrightarrow{y_1}$
\item $\overrightarrow{O_1O_2}=a\overrightarrow{x_1}+b\overrightarrow{y_1}$
\end{itemize}
\end{minipage}
\begin{minipage}{15em}
\vspace{2\baselineskip}
\begin{align*}
\overrightarrow{O_1M} &= \overrightarrow{O_1O_2} + \overrightarrow{O_2M} \\
x\overrightarrow{x_1}+y\overrightarrow{y_1} &=a\overrightarrow{x_1}+b\overrightarrow{y_1} + X\overrightarrow{x_1}+Y\overrightarrow{y_1}
\end{align*}
\end{minipage}
\end{multicols}
\vspace{1\baselineskip}


\begin{multicols}{2}
\begin{minipage}{23em}
Par projection sur $\overrightarrow{x_1}$  : \\
 $x = a+X$
\end{minipage}
\begin{minipage}{15em}
Par projection sur $\overrightarrow{y_1}$ 1 : \\
 $y = b+Y$
\end{minipage}
\end{multicols}



\item Exemple  : \\

On a $\overrightarrow{O_1M}:\coordo{2}{3}_{R_1}$ et $\overrightarrow{O_1O_2}:\coordo{1}{0}$. \\

On calcule alors $\overrightarrow{O_2M}:\coordo{2+1}{3+0}$

\item Matriciellement 

On dira simplement qu'on fait l'opération $V = T + U$.

\end{enumerate}

\newpage

\begin{Large} \item L'homothétie \end{Large} \\

\begin{enumerate}
\item Définition :  \\
On appelle homothétie de centre $O$ de rapport $r$ l'application qui a un point $M$ du plan associe le points $M'$ tel que $\overrightarrow{OM'} = r\overrightarrow{OM}$

\begin{multicols}{2}
\begin{minipage}{23em}
\includegraphics[width=2in]{image}
\end{minipage}
\begin{minipage}{23em}
\vspace{3\baselineskip}
En fait c'est un grossissement.
\end{minipage}
\end{multicols}

\item Exemple : \\
On a $\overrightarrow{OM}:\coordo{2}{3}$. 
L'image de $\overrightarrow{OM}$ par l'homothétie de centre $0$ de rapport $3$ est $\overrightarrow{OM'}:\coordo{6}{9}$.

\item Matriciellement : \\

On dira simplement qu'on fait l'opération $V = rU$.
\end{enumerate}

\hrulefill \\

\begin{Large} \item A quoi est-ce que ça sert ? \end{Large} \\


Dans de nombreux domaine il est utile d'opérer ces transformations et pour vous convaincre, prenons l'exemple de la robotique.

\begin{itemize}
\item On a un bras robotisé muni de 3 moteurs : épaule, coude, poignet.
\begin{multicols}{2}
\begin{minipage}{23em}
\includegraphics[width=3in]{image}
\end{minipage}
\begin{minipage}{23em}
\vspace{3\baselineskip}
Un capteur sur la main mesure les coordonnées de $\overrightarrow{O_6M}: \coordo{x_6}{y_6}_{R_6}$. \\

Pour obtenir les coordonnées de $\overrightarrow{O_0M}_{R_0}$, on doit faire les opérations suivantes :  \\
\end{minipage}
\end{multicols}
\end{itemize}



\end{enumerate}

\end{document}


